\documentclass[conference]{IEEEtran}
\IEEEoverridecommandlockouts
% The preceding line is only needed to identify funding in the first footnote. If that is unneeded, please comment it out.
\usepackage{cite}
\usepackage{amsmath,amssymb,amsfonts}
\usepackage{algorithmic}
\usepackage{graphicx}
\usepackage{textcomp}
\usepackage{xcolor}
\usepackage{float}

\floatstyle{boxed} 
\restylefloat{figure}

\def\BibTeX{{\rm B\kern-.05em{\sc i\kern-.025em b}\kern-.08em
    T\kern-.1667em\lower.7ex\hbox{E}\kern-.125emX}}
\begin{document}

\title{Workshop 2: System Design for Detecting Sleep States Using Accelerometers}

\author{
    \IEEEauthorblockN{Juan Carlos Quintero Rubiano}
    \IEEEauthorblockA{Code: 20232020172\\
    \textit{Systems Engineering} \\
    \textit{Francisco Jose de Caldas District University}\\
    Bogota, Colombia \\
    jcquineror@udistrital.edu.co}\\
    \IEEEauthorblockN{Juan Felipe Wilches Gomez}
    \IEEEauthorblockA{Code: 20231020137\\
    \textit{Systems Engineering} \\
    \textit{Francisco Jose de Caldas District University}\\
    Bogota, Colombia \\
    jfwilchesg@udistrital.edu.co}
    \and
    \IEEEauthorblockN{Juan Nicolas Diaz Salamanca}
    \IEEEauthorblockA{Code: 20232020059\\
    \textit{Systems Engineering} \\
    \textit{Francisco Jose de Caldas District University}\\
    Bogota, Colombia \\
    jndiazs@udistrital.edu.co}
    }

\maketitle

\begin{abstract}
    This document builds upon the systemic analysis conducted in Workshop 1 for the Kaggle competition "Child Mind Institute - Detect Sleep States." The focus is on designing a robust system to detect sleep states using accelerometer data. Key findings from the previous analysis, including constraints, data characteristics, and chaos-theory factors, are summarized to guide the design process. The proposed design aims to address identified weaknesses and optimize system performance.
    \end{abstract}
    
    \begin{IEEEkeywords}
    System design, sleep state detection, accelerometers, chaos theory, optimization.
    \end{IEEEkeywords}
    
    \section{Introduction}
    \subsection{Overview of Workshop 1 Findings}
    The systemic analysis conducted in Workshop 1 provided a comprehensive understanding of the system for detecting sleep states using accelerometer data. The following key findings were identified:
    
    \begin{itemize}
        \item \textbf{Constraints:} 
        \begin{itemize}
            \item Sleep periods must be at least 30 minutes long, with interruptions no longer than 30 minutes.
            \item Only the longest sleep window per night is recorded.
            \item Device removal periods are not annotated, introducing potential gaps in data.
        \end{itemize}
        \item \textbf{Data Characteristics:}
        \begin{itemize}
            \item The dataset includes accelerometer data with features such as \texttt{anglez} and \texttt{enmo}, which are critical for detecting sleep states.
            \item Labels for sleep onset and wake events are provided, enabling supervised learning approaches.
        \end{itemize}
        \item \textbf{Chaos-Theory Factors:}
        \begin{itemize}
            \item Sensitivity to initial conditions: Small variations in accelerometer data can lead to significant changes in sleep state classification.
            \item Randomness in sleep patterns: External factors such as environmental conditions and individual differences introduce unpredictability.
        \end{itemize}
    \end{itemize}

    \end{document}